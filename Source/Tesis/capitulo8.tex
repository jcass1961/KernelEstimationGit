%% Los cap'itulos inician con \chapter{T'itulo}, estos aparecen numerados y
%% se incluyen en el 'indice general.
%%
%% Recuerda que aqu'i ya puedes escribir acentos como: 'a, 'e, 'i, etc.
%% La letra n con tilde es: 'n.

\chapter{Conclusiones}
\label{Conclusiones}

Propusimos un nuevo estimador para el parámetro de textura de la distribución $\mathcal{G}_I^0$ basado en la minimización de una distancia estocástica entre el modelo (que depende de dicho parámetro) y una estimación de la función de densidad subyacente utilizando núcleos asimétricos.
Definimos tres modelos de contaminación inspirados en situaciones reales para evaluar el impacto de los valores atípicos en el desempeño de los estimadores.
Las dificultades de estimar el parámetro de textura de la distribución $\mathcal{G}_I^0$ se investigaron más a fondo y se justificaron a través de nuevos resultados teóricos que indican que esta distribución tiene colas pesadas.

Con respecto al impacto de la contaminación en el rendimiento de los estimadores, observamos lo siguiente.
Los estimadores de Momento fraccional y Logcumulantes presentaron problemas en la convergenciar.

\begin {itemize}
	\item Caso 1: Independientemente de la intensidad de la contaminación, cuanto mayor sea el número de looks, menor será el porcentaje de situaciones para las que no se logra convergencia.
	\item Casos 2 y 3: el porcentaje de situaciones para las que no hay convergencia aumenta con el nivel de contaminación, y se reduce con $\alpha$.
\end {itemize}

En el caso de un solo look, el estimador propuesto no presenta resultados excelentes, pero nunca deja de converger, cosa que puede suceder con los otros estimadores.

El nuevo estimador presenta buenas propiedades medidas por su sesgo y error cuadrático medio. Es competitivo con los MV estimadores, momento fraccionario y Logcumulant en situaciones sin contaminación, y supera a las otras técnicas incluso en presencia de pequeños niveles de contaminación.

Por esta razón, sería aconsejable usar $\widehat{\alpha}_{\text{T}} $ en cada situación, especialmente cuando se usan muestras pequeñas y/o cuando existe la posibilidad de tener datos contaminados.
El costo computacional adicional incurrido al usar este estimador es, como máximo, veinte veces mayor que el requerido para calcular $\widehat{\alpha}_{\text{ML}}$, pero sus ventajas son más importantes que este aumento en el tiempo de procesamiento.

\section{Apéndice}

% \ section {Información computacional}

Las simulaciones se realizaron utilizando el lenguaje y el entorno \texttt R para computación estadística~\cite{RLanguage}.
La función \texttt{adaptIntegrate} del paquete \texttt{cubature} se utilizó para realizar la integración numérica requerida para evaluar la distancia triangular, el algoritmo utilizado es una integración multidimensional adaptativa sobre hipercubos. Para encontrar numéricamente $\widehat\alpha_{\text{LC}}$ utilizamos la función `` uniroot '' implementada en \texttt R.

Una parte de las simulaciones se realizó en una plataforma compuesta por un Intel(R) Core i7 con $8$ GB de memoria y $64$ bits  Windows  7. Otra parte se hizo en un equipo similar pero con $16$ GB de memoria RAM.
Los códigos y los datos están disponibles a solicitud del autor correspondiente.