\section{Demostración de algunos resultados}

\subsection{Demostración de las ecuaciones dadas en~\eqref{MISEoptLN}}
\begin{dem}
	De acuerdo a~\cite{Libnegue2013} expresiones para el sesgo, la varianza de $\widehat{f}_n$ para le caso de núcleo Lognormal son:
	\begin{align}
	Sesgo(\widehat{f}_n)&=x(e^{3 b^2}-1) f'(x)+\dfrac{1}{2}\{x^2(e^{3 b^2}-1)+x^2 e^{3b^2}(e^{b^2}-1)\}f''(x)+o(b^2)\\
	Var(\widehat{f}_n)&=\dfrac{1}{2 n x b \sqrt{\pi}} f(x)+o(n^{-1}b^{-1})
	\end{align}
	
	Haciendo un desarrollo de Taylor de orden 4 del error cuadrático medio ECM\eqref{ErrorCuadraticoMedio} alrededor de $b=0$ obtenemos que
	\begin{align}
	ECM(\widehat{f}_n)=&\frac{f(x)}{x \sqrt{\pi}}\left[\frac{1}{2 n b}-\frac{3 b}{8 n}+\frac{9 b^3}{64 n}\right]+\\
	                  &\frac{1}{4} b^4 x^2 \left(x^2 f''(x)^2+9 f'(x)^2+6 x f'(x) f''(x)\right)+O\left(b^5\right)\\
	\end{align}
	
	Despreciando los términos de mayor orden tenemos que
	\begin{align}
	MISE(\widehat{f}_n)\approx&\frac{f(x)}{x \sqrt{\pi}}\left[\frac{1}{2 b n}-\frac{3 b}{8 n}+\frac{9 b^3}{64 n}\right]+\frac{1}{4} b^4 x^2\left(3f'(x)+xf''(x)\right)^2
	\end{align}
	 
	
	
\end{dem}


%\begin{lemma}
%	\begin{align}
%	\int_0^{+\infty} dK_{\Gamma^1_{(x/b+1),b}}(t)=O(b^{-1})
%	\end{align}
%\end{lemma}
%
%\begin{lemma}
%	\begin{align}
%	\int_0^{+\infty} dK_{{\text{\tiny LN}}_{\left(log(x)+b^2,b \right)}}(t) & =O(b^{-2})
%	\end{align}
%\end{lemma}
%
%\begin{dem}
%	\begin{align}
%	K_{{\text{\tiny LN}}_{\left(log(x)+b^2,b \right)}}(t) & =\frac{1}{t \sqrt{2 \pi} b} \exp\Big\{-\frac{\left(\log t - \log x -b^2\right)^2}{2b^2}\Big\},
%	\end{align}
%	Entonces, la derivada de $K_{{\text{\tiny LN}}_{\left(log(x)+b^2,b \right)}}(t)$ respecto de $t$ es
%	\begin{align}
%	dK_{{\text{\tiny LN}}_{\left(log(x)+b^2,b \right)}}(t) & =\frac{(\log (x)-\log (u)) e^{-\frac{\left(h^2-\log (u)+\log (x)\right)^2}{2 h^2}}}{\sqrt{2 \pi } h^3 u^2}
%	\end{align}
%\end{dem}
%Entonces
%	
%\begin{align}
%\int_0^{+\infty} \mid dK_{{\text{\tiny LN}}_{\left(log(x)+b^2,b \right)}}(t) \mid dt&=\int_0^{x} \frac{(\log (x)-\log (u)) e^{-\frac{\left(h^2-\log (u)+\log (x)\right)^2}{2 h^2}}}{\sqrt{2 \pi } h^3 u^2} dt\\
%& + \int_x^{+\infty} - \frac{(\log (x)-\log (u)) e^{-\frac{\left(h^2-\log (u)+\log (x)\right)^2}{2 h^2}}}{\sqrt{2 \pi } h^3 u^2} dt\\
%&=2 \frac{e^{-\frac{h^2}{2}}}{\sqrt{2 \pi } h x}
%&\leq c_1(x)h^{-1}
%\end{align}
%donde $c_1(x)=\dfrac{\sqrt{2}}{\sqrt{\pi } x}$ es una función integrable sobre compactos $I \subset (0,+\infty))$.