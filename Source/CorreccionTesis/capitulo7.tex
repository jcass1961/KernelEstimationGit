%% Los cap'itulos inician con \chapter{T'itulo}, estos aparecen numerados y
%% se incluyen en el 'indice general.
%%
%% Recuerda que aqu'i ya puedes escribir acentos como: 'a, 'e, 'i, etc.
%% La letra n con tilde es: 'n.

\chapter{Resultados Teóricos}
\label{ResultadosTeoricos}

%Los estimadores de mínima distancia (MDE) surgen como una alternativa más robusta a los estimadores paramétricos clásicos, como Máxima Verosimilitud (MV). La idea de la cual parte resulta muy natural, es encontrar el valor del vector de parámetros que minimiza la distancia entre el modelo teórico y aquel que resulta de la información muestral.
%
%%Las preguntas que nos podemos hacer es 
%%\begin{itemize}
%%	\item ¿Cómo modelamos los datos muestrales?
%%	\item ¿Qué medida de distancia considerar?
%%\end{itemize}
%%
%%Para responder a la primera pregunta apelamos a estimar la función de densidad subyacente con núcleos asimétricos. Para responder la segunda pregunta relajamos el concepto de métrica estudiando otras medidas como las distancias estocásticas, como forma de medir la discrepancia entre dos funciones de densidad. Elegimos la distancia triangular porque, de las medidas que evaluamos, ésta fue la que mejor performance mostró. En el capítulo~  \ref{ResultadosEmpiricos} hicimos un estudio Monte Carlo para evaluar la perfomance del estimador propuesto. Vimos que para datos sin contaminación la propuesta tiene un comportamiento similar al estimador MV, mientras que bajo contaminación la mejora en algunos casos estudiados.
%
%En general, los MDE estimadores se definen midiendo la discrepancia entre funciones de distribución. Wolfowitz \citet{Wolfowitz1954}, Parr and Schucany \citet{Parr1980} son algunas de las principales referencias. Como indican Cao et al. \citet{cao1995minimum} en los casos más usuales las funciones de distribución son absolutamente continuas. Ellos aplican los resultados de Parr and Schucany \citet{parr1982} al caso donde se miden distancias entre funciones de densidad. Consideran un estimador por núcleos simétricos para estimar la función de densidad subyacente en vez de utilizar la función de distribución empírica.

En este capítulo probaremos la convergencia fuerte del estimador propuesto en esta tesis. Se usarán algunos resultados probados anteriormente, entre ellos la convergencia uniforme sobre conjuntos compactos de $f_{\mathcal{G}_I^0(\alpha,-\alpha-1,L)}$ cuando $\alpha \to -1$ y $\alpha \to -\infty$ visto en las proposiciones~\ref{conv1} y~\ref{conv2} respectivamente, y resultados sobre convergencia fuerte en norma $L^1$ mencionado en en el teorema~\ref{ConvFuerte_fn}. Primeramente daremos algunas definiciones.

Sea $f_{\mathcal{G}_I^0(\alpha,\gamma,L)}$ la función de densidad correspondiente al modelo $\mathcal{G}_I^0$ como la definida en la sección~\ref{ModeloGI0}. Recordemos que el espacio paramétrico está formado por los valores de $L\geq 1$, $\gamma>0$ y $\alpha<-1$ para que exista primer momento finito. 
Asumimos $L$ conocido y $\gamma^* = -\alpha-1$, de esta manera estamos considerando un modelo reducido a un parámetro. La función de densidad para este caso la llamaremos $f_{\alpha}$ y queda definida como
\begin{equation}
f_{\alpha}( z) =\frac{L^{L}\Gamma ( L-\alpha
	) }{(-\alpha-1) ^{\alpha }\Gamma ( -\alpha ) \Gamma (
	L) }\cdot  
\frac{z^{L-1}}{((-\alpha-1) +zL) ^{L-\alpha }}.
\label{modeloreducido}
\end{equation}

Sea $\widehat{\alpha}_n$ el estimador del parámetro $\alpha$ definido por
\begin{align}
\label{EstMinDist}
\begin{split}
\widehat{\alpha}_{n}=\arg\min\limits_{\alpha \in (-\infty,-1)}d(f_{\alpha},\widehat{f}_n)=\arg\min\limits_{\alpha \in (-\infty,-1)}D(f_{\alpha},\widehat{f}_n),
\end{split}
\end{align}
donde $\widehat{f}_n$ es un estimador de la función de densidad subyacente usando núcleos asimétricos como los estudiados en la sección~\ref{NucleosAsimetricos}, $d$ es la distancia triangular definida en la sección~\ref{dist} y $D=d^{1/2}$ es una métrica de acuerdo a la proposición~\ref{DTdistancia}. Observemos que $D$ es una transformación monótona de $d$ lo que justifica las igualdades dadas en la ecuación~\eqref{EstMinDist}.

En algunos casos puede suceder que el mínimo definido en~\eqref{EstMinDist} no exista o que, por cuestiones numéricas no se puede alcanzar. Entonces, para evitar este tipo de problemas, esta definición se puede relajar definiendo $\widehat{\alpha}_n$ como cualquier valor que satisface
\begin{align}
\label{EstMinDistRelajado}
d(f_{\widehat{\alpha}_{n}},\widehat{f}_n) \leq \inf\limits_{\alpha \in (-\infty,-1)}d(f_{{\alpha}},\widehat{f}_n) + k_n.
\end{align}
donde $k_n$ es una sucesión que tiende a cero cuando $n$ tiende a infinito. 

%basándonos en los resultados \citet{cao1995minimum,parr1982}.
\vspace{1cm}

En lo que sigue vamos a considerar:
\begin{itemize}
	\item $\mathcal{F}=\{f \in L^1(0,+\infty) : f \geq 0\}$.
	\item $D(f,g)=d^{1/2}(f,g)$ y $f,g \in \mathcal{F} $ que, de acuerdo a la proposición~\ref{DTdistancia}, resulta una distancia en $\mathcal{F}$.
	\item $\{\widehat{\alpha}_n\}_{n=1}^{\infty}$ los estimadores de mínima distancia dados en la definición~\eqref{EstMinDistRelajado}.
	\item $\widehat{f}_n$ un estimador por núcleos asimétricos de la función de densidad subyacente dado en la definición~\ref{DefNucleo} y que cumple las condiciones del teorema~\ref{ConvergenciaFuerte}.
\end{itemize}

\section{Consistencia fuerte de $\widehat{\alpha}_{n}$}

\vspace{0.5cm}

%-------------------------------------------------------------------------------------------------------------------------\\
Para probar la convergencia fuerte del estimador definido en~\eqref{EstMinDistRelajado} seguiremos las ideas de \citet{parr1982}. Daremos primero algunos resultados previos.
%Por otro lado, la condición (ii) en~  \ref{TeoParr} se verifica, en el caso que la discrepancia considerada sea una distancia.

\begin{proposition}
	\label{continuidad}
	La aplicación del intervalo $(-\infty,-1)$ en $L^{1}(0,\infty)$ dada por
	$\alpha \mapsto f_{\alpha}$ es continua.
\end{proposition}
\begin{dem}
	Por la continuidad de $f_{\alpha}(z)$ en $\alpha$, si $\alpha_{n} \to \alpha$, 
	entonces $f_{\alpha_{n}}(z) \to f_{\alpha}(z)$ para todo $z>0$.
	Siendo $f_{\alpha_{n}}$ y $f_{\alpha}$ densidades de probabilidad, por el Teorema de \citet{scheffe1947}, obtenemos
	la convergencia en $L^{1}(0,\infty)$.
\end{dem}

\begin{proposition}
	\label{CondivParr}
	Sea $\alpha_{*} \in (-\infty,-1)$ fijo, si la sucesión $\{\alpha_{n}\}_{n \ge 1} \subset (-\infty,-1)$ verifica
	$\lim\limits_{n \to \infty} d(f_{\alpha_*}, f_{\alpha_{n}}) = 0$ siendo $d$ la distancia estocástica definida en~\eqref{triangular},
	%  \begin{align*}
	%   \lim_{n \to \infty} d_{T}(f_{\alpha_{*}}, f_{\alpha_{n}}) = 0,
	%  \end{align*}
	entonces $\alpha_{n} \to \alpha_{*}$.
\end{proposition}
%\begin{dem}
%	Supongamos que $\alpha_{n} \not\to \alpha_{*}$, vamos a probar que existe una subsucesión $\{\alpha_{n_{k}}\}_{k \ge 1}$
%	tal que $d(f_{\alpha_{*}}, f_{\alpha_{n_{k}}}) \not\to 0$, lo que representa una contradicción.
%	Como $\alpha_{n} \not\to \alpha_{*}$, existe una subsucesión $\{\alpha_{n_{k}}\}_{k \ge 1}$
%	que verifica alguna de las siguientes afirmaciones:
%	\begin{itemize}
%		\item $\alpha_{n_{k}} \to \alpha_{0}$, donde $\alpha_{0} \in (-\infty,-1)$ y $\alpha_{0} \ne \alpha_{*}$.
%		\item $\alpha_{n_{k}} \to -1$, 
%		\item $\alpha_{n_{k}} \to -\infty$.
%	\end{itemize}
%	En el primer caso, como consecuencia de la proposición~\ref{continuidad} que muestra la continuidad respecto al parámetro $\alpha$, se verifica
%	\begin{align*}
%	\lim_{k\to\infty} d(f_{\alpha_{*}}, f_{\alpha_{n_{k}}}) = d(f_{\alpha_{*}}, f_{\alpha_{0}}) > 0.
%	\end{align*}
%	En el caso siguiente, por la proposición \ref{conv2} tenemos que $f_{\alpha_{n_{k}}} \to 0$ uniformemente
%	en $[z_{1},z_{2}]$, por lo tanto
%	\begin{align*}
%	\liminf_{k\to\infty} d\pa{f_{\alpha_{*}}, f_{\alpha_{n_{k}}}} 
%	& \ge \liminf_{k\to\infty}  \int_{z_{1}}^{z_{2}}\frac{(f_{\alpha_{*}}\pa{z}-f_{\alpha_{n_{k}}}\pa{z})^{2}}
%	{f_{\alpha_{*}}\pa{z}+f_{\alpha_{n_{k}}}\pa{z}}dz \\
%	& = \int_{z_{1}}^{z_{2}}f_{\alpha_{*}}\pa{z}dz >0.
%	\end{align*}
%	Por último, siendo que $f_{\alpha_{*}}\ne f_{\Gamma(L,L)}$ donde $f_{\Gamma(L,L)}$ es la función de densidad del modelo $\Gamma(L,L)$, existe $[z_{1},z_{2}]\subset\pa{0,\infty}$, tal que
%	\begin{align*}
%	\int_{z_{1}}^{z_{2}}\frac{\pa{f_{\alpha}\pa{z}-f_{\Gamma(L,L)}\pa{z}}^{2}}
%	{f_{\alpha}\pa{z}+f_{\Gamma(L,L)}\pa{z}}dz >0,
%	\end{align*}
%	como $f_{\alpha_{n_{k}}} \to f_{\Gamma(L,L)}$ uniformemente en $[z_{1},z_{2}]$, el
%	\begin{align*}
%	\liminf_{k\to\infty} d\pa{f_{\alpha_{*}}, f_{\alpha_{n_{k}}}} 
%	& \ge \liminf_{k\to\infty}  \int_{z_{1}}^{z_{2}}\frac{(f_{\alpha_{*}}\pa{z}-f_{\alpha_{n_{k}}}\pa{z})^{2}}
%	{f_{\alpha_{*}}\pa{z} + f_{\alpha_{n_{k}}}\pa{z}}dz \\
%	& = \int_{z_{1}}^{z_{2}}\frac{\pa{f_{\alpha}\pa{z}-f_{\Gamma(L,L)}\pa{z}}^{2}}
%	{f_{\alpha}\pa{z}+f_{\Gamma(L,L)}\pa{z}}dz >0.
%	\end{align*}
%	Esto muestra que $d(f_{\alpha_{*}}, f_{\alpha_{n_{k}}}) \not\to 0$.
%	La contradicción proviene de suponer que $\alpha_{n} \not\to \alpha_{*}$.
%\end{dem}
\begin{dem} \ 
	\begin{itemize}
		\item Sea $\{\alpha_{n}\}_{n \ge 1} \subset (-\infty,-1)$. Entonces se verifican alguna de estas tres afirmaciones:
		
		\begin{enumerate}[(i)]
			\item\label{it: -1} $\sup\{\alpha_{n}: n \ge 1\} = -1$.
			\item\label{it: -inf} $\inf\{\alpha_{n}: n \ge 1\} = -\infty$.
			\item\label{it: acotada}  existe un conjunto compacto $I \subset (-\infty,-1)$ tal que $\{\alpha_{n}\}_{n \ge 1} \subset I$, y por lo tanto $\{\alpha_{n}\}_{n \ge 1}$ está acotada.
			%	existen $a \text{ y } b \in (-\infty,-1)$ tal que $-\infty < a <\inf\{\alpha_{n}: n \ge 1\}$ y $\sup\{\alpha_{n}: n \ge 1\} < b < -1$
			%	tales que el intervalo compacto $I = [a, b] \subset (-\infty,-1)$ contiene a la sucesión $\{\alpha_{n}\}_{n \ge 1}$.
			%por lo tanto existe $\alpha_{0} \in I$ y una subsucesión $\{\alpha_{n_{k}}\}_{k \ge 1}$ que converge a $\alpha_{0}$. 
		\end{enumerate}
		
		Probemos que no se verifica ni~\ref{it: -1}) ni~\ref{it: -inf}).
		\begin{enumerate}[(i)]
			\item Supongamos que  $\sup\{\alpha_{n}: n \ge 1\} = -1$. 
			
			Entonces existe una subsucesión $\alpha_{n_{k}} \to -1$ cuando $k \to +\infty$. Por la proposición~\ref{conv2} tenemos que $f_{\alpha_{n_{k}}} \to 0$ uniformemente
			en $[z_{1},z_{2}]$, por lo tanto
			\begin{align*}
			0=\lim_{k\to\infty} d\pa{f_{\alpha_{*}}, f_{\alpha_{n_{k}}}} 
			& \ge \lim_{k\to\infty}  \int_{z_{1}}^{z_{2}}\frac{(f_{\alpha_{*}}\pa{z}-f_{\alpha_{n_{k}}}\pa{z})^{2}}
			{f_{\alpha_{*}}\pa{z}+f_{\alpha_{n_{k}}}\pa{z}}dz \\
			& = \int_{z_{1}}^{z_{2}}f_{\alpha_{*}}\pa{z}dz >0.
			\end{align*}
			Lo que resulta una contradicción.
			
			\item Supongamos que $\inf\{\alpha_{n}: n \ge 1\} = -\infty$.
			
			Entonces, existe una subsucesión $\alpha_{n_{k}} \to -\infty$ cuando $k \to +\infty$. Dado que $f_{\alpha_{*}}\ne f_{\Gamma(L,L)}$ donde $f_{\Gamma(L,L)}$ es la función de densidad del modelo $\Gamma(L,L)$, existe $[z_{1},z_{2}]\subset\pa{0,\infty}$, tal que
			\begin{align*}
			\int_{z_{1}}^{z_{2}}\frac{\pa{f_{\alpha_{*}}\pa{z}-f_{\Gamma(L,L)}\pa{z}}^{2}}
			{f_{\alpha_{*}}\pa{z}+f_{\Gamma(L,L)}\pa{z}}dz >0,
			\end{align*}
			como $f_{\alpha_{n_{k}}} \to f_{\Gamma(L,L)}$ uniformemente en $[z_{1},z_{2}]$ por la proposición~\ref{conv1}, 
			\begin{align*}
			0=\lim_{k\to\infty} d\pa{f_{\alpha_{*}}, f_{\alpha_{n_{k}}}} 
			& \ge \lim_{k\to\infty}  \int_{z_{1}}^{z_{2}}\frac{(f_{\alpha_{*}}\pa{z}-f_{\alpha_{n_{k}}}\pa{z})^{2}}
			{f_{\alpha_{*}}\pa{z} + f_{\alpha_{n_{k}}}\pa{z}}dz \\
			& = \int_{z_{1}}^{z_{2}}\frac{\pa{f_{\alpha_{*}}\pa{z}-f_{\Gamma(L,L)}\pa{z}}^{2}}
			{f_{\alpha_{*}}\pa{z}+f_{\Gamma(L,L)}\pa{z}}dz >0.
			\end{align*}
			Lo que resulta una contradicción.
		\end{enumerate}
		Entonces existe un conjunto compacto $I \subset (-\infty,-1)$ tal que $\{\alpha_{n}\} \subset I$.
	\end{itemize}
	
	\begin{itemize}
		\item Probemos que $\alpha_*$ es el único punto de acumulación de $\{\alpha_{n}\}_{n \ge 1}$.
	\end{itemize}
	
	Como $\{\alpha_{n}\}_{n \ge 1}$ es una sucesión acotada tiene un punto de acumulación, digamos $\alpha_0$. Es decir, existe una subsucesión $\{\alpha_{n_{k}}\} \subset I$ tal que $\alpha_{n_{k}} \to \alpha_{0}$ cuando $k \to +\infty$. Entonces por la proposición~\ref{continuidad} $\| f_{\alpha_{n_{k}}} - f_{\alpha_{0}}\|_{1} \to 0$ cuando $k \to +\infty$, luego por~\eqref{AcotacionDTporL1} $d(f_{\alpha_{n_k}},f_{\alpha_0}) \to 0$. Como por hipótesis $d(f_{\alpha_{n}},f_{\alpha_*}) \to 0$ resulta que $d(f_{\alpha_{n_k}},f_{\alpha_*}) \to 0$, por lo tanto, $f_{\alpha_0} = f_{\alpha_*}$. Entonces, por la identificabilidad del modelo, resulta que $\alpha_0=\alpha_*$.
	
	Si existiera otro punto de acumulación $\tilde{\alpha}_0$ aplicando el mismo razonamiento llegaríamos a que $\tilde{\alpha}_0=\alpha_*$.
	
	%    \begin{itemize}
	%    	\item Probemos que $\alpha_*$ es un único punto de acumulación.
	%    \end{itemize}
	%	Supongamos que existe un $\tilde{\alpha}_*$ punto de acumulación de $\{\alpha_{n}\}_{n \ge 1}$ con $\alpha_* \neq \tilde{\alpha}_*$. Entonces existe una una subsucesión $\{\tilde{\alpha}_{n_{k}}\} \subset I$ tal que $\{\tilde{\alpha}_{n_{k}}\} \to \tilde{\alpha}_*$ cuando $k \to +\infty$. Aplicando el mismo razonamiento que en el punto anterior llegamos a que $\tilde{\alpha}_*=\alpha_*$, lo que es una contradicción.
	
	Luego $\{\alpha_{n}\}_{n \ge 1}$ es una sucesión acotada que tiene un único punto de acumulación, entonces $\alpha_{n} \to \alpha_*$ cuando $n \to +\infty$ que es lo que queríamos probar.
\end{dem}

\begin{theorem}
	Sea $Z_1, \ldots, Z_n$ una sucesión de variables aleatorias iid donde $Z_i \sim f_{\alpha_{*}}$ con $f_{\alpha_{*}}$ la función de densidad definida en~\eqref{modeloreducido} y sea $\widehat{f}_n$ un estimador por núcleos asimétricos de la función de densidad subyacente que cumple las condiciones del teorema~\ref{ConvergenciaFuerte}. Sea $\{\widehat{\alpha}_n\}$ una sucesión MDE estimadores como los definidos en~\eqref{EstMinDistRelajado}.
	
	Si $\lim\limits_{n \to \infty} \dfrac{n b^{2r_1}}{\ln{n}}  = +\infty $ con $r_1$ el establecido en cada caso de acuerdo al núcleo elegido, entonces $\widehat{\alpha}_n \stackrel{cs}{\longrightarrow} \alpha_{*}$ cuando $n \longrightarrow +\infty.$
\end{theorem}

\begin{dem}
	
	Observemos que basta probar que $D(f_{\alpha_{*}},f_{\widehat{\alpha}_n}) \stackrel{cs}{\longrightarrow} 0$ cuando $n \rightarrow +\infty$, porque utilizando la proposición~\ref{CondivParr} habremos probado que $\widehat{\alpha}_n \stackrel{cs}{\longrightarrow} \alpha_{*}.$
	
    Por la desigualdad triangular se cumple que
	\begin{align}
	D(f_{\alpha_{*}},f_{\widehat{\alpha}_n}) \leq D(f_{\alpha_{*}},\widehat{f}_n)+D(\widehat{f}_n,f_{\widehat{\alpha}_n}).
	\end{align}
\begin{itemize}
	\item Probemos que $D(f_{\alpha_{*}},\widehat{f}_n) \stackrel{cs}{\longrightarrow} 0$ cuando $n \to +\infty$.
\end{itemize}
	
	Por hipótesis $\lim\limits_{n \to \infty} \dfrac{n b^{2r_1}}{\ln{n}}  = +\infty $, entonces de acuerdo al teorema~\ref{ConvergenciaFuerte} $\| f_{\alpha_{*}} -\widehat{f}_n\|_1 \stackrel{cs}{\longrightarrow} 0$ cuando $n \to +\infty$. Recordando que por~\eqref{AcotacionDTporL1} $d(f_{\alpha_{*}},\widehat{f}_n) \leq \| f_{\alpha_{*}} -\widehat{f}_n\|_1$ entonces $D(f_{\alpha_{*}},\widehat{f}_n)=d^{1/2}(f_{\alpha_{*}},\widehat{f}_n) \stackrel{cs}{\longrightarrow} 0$ cuando $n \to +\infty$. 

\begin{itemize}
	\item Probemos que $D(\widehat{f}_n,f_{\widehat{\alpha}_n}) \stackrel{cs}{\longrightarrow} 0$ cuando $n \to +\infty$.
\end{itemize}

Por definición de ínfimo %y por el teorema~\ref{ConvergenciaFuerte}
\begin{align*}
0 \leq \inf_{\alpha\in I} D{(\widehat{f}_{n},f_{\alpha}}) \leq  D(\widehat{f}_{n},f_{\alpha_{*}})\stackrel{cs}{\longrightarrow} 0
\end{align*}
cuando $n\to +\infty$.
Por la definición~\ref{EstMinDistRelajado}
\begin{align}
\label{EstMinDistRelajado2}
\inf\limits_{\alpha \in I}D(f_{{\alpha}},\widehat{f}_n) \leq D(f_{\widehat{\alpha}_{n}},\widehat{f}_n) \leq \inf\limits_{\alpha \in I}D(f_{{\alpha}},\widehat{f}_n) + k_n,
\end{align}
por lo tanto, $D(f_{\widehat{\alpha}_{n}},\widehat{f}_n) \stackrel{cs}{\longrightarrow} 0$ cuando $n \to +\infty$, que es lo que queríamos probar.
\end{dem}

%\section{Demostración de algunos resultados}
%
%\subsection{Demostración de las ecuaciones dadas en~\eqref{MISEoptLN}}
%\begin{dem}
%	De acuerdo a \citet{Libnegue2013} expresiones para el sesgo, la varianza de $\widehat{f}_n$ para le caso de núcleo Lognormal son:
%	\begin{align}
%	Sesgo(\widehat{f}_n)&=x(e^{3 b^2}-1) f'(x)+\dfrac{1}{2}\{x^2(e^{3 b^2}-1)+x^2 e^{3b^2}(e^{b^2}-1)\}f''(x)+o(b^2)\\
%	Var(\widehat{f}_n)&=\dfrac{1}{2 n x b \sqrt{\pi}} f(x)+o(n^{-1}b^{-1})
%	\end{align}
%	
%	Haciendo un desarrollo de Taylor de orden 4 del error cuadrático medio ECM\eqref{ErrorCuadraticoMedio} alrededor de $b=0$ obtenemos que
%	\begin{align}
%	ECM(\widehat{f}_n)=&\frac{f(x)}{x \sqrt{\pi}}\left[\frac{1}{2 n b}-\frac{3 b}{8 n}+\frac{9 b^3}{64 n}\right]+\\
%	&\frac{1}{4} b^4 x^2 \left(x^2 f''(x)^2+9 f'(x)^2+6 x f'(x) f''(x)\right)+o\left(b^4+n^{-1}b^{-1}\right)\\
%	\end{align}
%	
%	Si $n^{-1}b^{-1} \to 0$ cuando $n \to +\infty$ entonces $\dfrac{b}{n}$ y $\dfrac{b^3}{n}$ son $o(n^{-1}b^{-1})$.
%	Luego
%	\begin{align}
%	\label{MISE}
%	MISE(\widehat{f}_n)=&\dfrac{1}{2 n b \sqrt{\pi}} \int_{0}^{+\infty}\frac{f(x)}{x}+\frac{b^4}{4} \int_{0}^{+\infty} x^2\left(3f'(x)+xf''(x)\right)^2+o\left(b^4+n^{-1}b^{-1}\right).
%	\end{align}
%	
%	El ancho de banda $b$ que minimiza los términos principales en~\ref{MISE} es 
%	
%	\begin{align}
%		b_{\small{\text{LN}}}_{*}=\dfrac{\left[\dfrac{1}{2\sqrt{\pi}}\displaystyle{\int_0^{+\infty}}\frac{f(x)}{x}\right]^{1/5}}{\left[\displaystyle{\int_0^{+\infty}} x^2(3f'(x)+xf''(x))^2\right]^{1/5}}n^{-1/5}.
%	\end{align}
%	y el valor del MISE óptimo es
%	\begin{align}
%	\text{MISE}_{*}_{\text{\tiny{LN}}}=&\dfrac{5}{4}  \left[\dfrac{1}{2\sqrt{\pi}}\displaystyle{\int_{0}^{\infty}} \frac{f(x)}{x} \right]^{4/5} \\
%	&\times \left[\displaystyle{\int_{0}^{\infty}}  x^2 (3 f'(x) + x f''(x))^2 \right]^{1/5} n^{-4/5} 
%\end{align}
%\end{dem}