\chapter{Información computacional}

% \ section {Información computacional}

Las simulaciones se realizaron utilizando el lenguaje y el entorno \texttt R para computación estadística~\cite{RLanguage}.
La función \texttt{adaptIntegrate} del paquete \texttt{cubature} se utilizó para realizar la integración numérica requerida para evaluar la distancia triangular, el algoritmo utilizado es una integración multidimensional adaptativa sobre hipercubos. Para encontrar numéricamente $\widehat\alpha_{\text{LC}}$ utilizamos la función \textit{uniroot} implementada en \texttt R.

Una parte de las simulaciones se realizó en una plataforma compuesta por un Intel(R) Core i7 con $8$ GB de memoria y $64$ bits  Windows  7. Otra parte se hizo en un equipo similar pero con $16$ GB de memoria RAM.
Los códigos y los datos están disponibles a solicitud del autor correspondiente.